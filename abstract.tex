\documentclass[a4paper]{article}

\usepackage[english]{babel}
\usepackage[utf8]{inputenc}
\usepackage{amsmath}
\usepackage{graphicx}
\usepackage[colorinlistoftodos]{todonotes}

\title{Your Paper}

\author{You}

\date{\today}

\begin{document}
\maketitle

\begin{abstract}
Your abstract.
\end{abstract}

\section{3-Hand manipulation}

\subsection{Introduction}

This paper presents a tool that allows to 2 or 3 users to manipulate virtual the same virtual object simultaneously. The 6 freedom degrees can be reproduced.

\subsection{State of the art}

Multi-User interaction tools already exist, but they are limited and not representative of the real world interactions. Some aren't symmetric: the DOF are distributed over the users, like "trackball". Others don't allow one of the DOF (the rotation around the axis defined by two handling hands) or at least don't reproduce it in a realistic way, like "SkeweR". By the way, the good ways to combine two interactions change from an application to another (ex : adding, compute the mean).

The determination of the sixth DOF is impossible because of the number of contact points, which is only 2. Therefore, it's impossible to generate forces able to make the object rotate around the defined axis.

\subsection{Solution}

The simple solution is to define a third point of handling. Now we have a plane and therefore a possible change of orientation. This rotation is computed as the combination of two (sufficient) rotations.

\section{Exchange of avatars}

\subsection{Introduction}

This paper presents and compares several ways to represent an exchange of avatars, for the actors populating the virtual environment to be aware of the exchange. Two cases are considered: if the user to be notified is involved in the exchange or not. One of the conclusions of the experiments is that it's not necessary to perform a visual feedback of the exchange even if it's more efficient.

\subsection{State of the art}

Exchange of avatar already has a lot of applications. In video games, a player can enhance his game experience by changing of character. In collaborative training, switching roles allows a trainee to understand the work as a whole and reinforces the team cohesion. Other researches studied the virtual body takeover sensation, which must be taken into account when you change your virtual body. However, the perception of the exchange in itself has not be studied before this paper.

\subsection{Solution}

\subsubsection{Formalisation}
As it was already explained, two points of view are considered: the point of view of a user witnessing the exchange and the one of a user experiencing it. For both cases, the interest of being aware of the exchange is obvious, since all users are likely to interact with each others and as a consequence being able to know which user is using which avatar.

For each exchange, the user experience was evaluated through two criteria:
\begin{description}
	\item[Understanding:] how difficult it is for the user to be aware of the exchange, in terms of time, feelings and errors (how many time a user refers to an avatar $A$ believing it is owned by the user 1 whereas it is owned by the user 2).
	\item[Perception:] how pleasant the exchange is for the user.
\end{description}

\subsubsection{Evaluation}

\paragraph{Witnessing exchange test}
\begin{description}
	\item[General description:] a user is placed in a virtual environment in which he has to perform some tasks. Sometimes he must call a specific expert among several. The experts regularly switch their avatars. Therefore, the user can make mistakes and call an expert instead of another. The goal of the experience is to find the best metaphor to represent a switching of avatars. The owners have to be identifiable during the switching, whatever the metaphor. In this experience, they are identified by a color.
	\item[Tested parameters:] 3 metaphors representing the switching of two avatar owners ; \textit{Flickering}, \textit{Ghost} and \textit{Popup notification}. \textit{Flickering} consists of making each avatar texture blink, changing temporarily its color to the one representing the new owner. \textit{Ghost} consists of making appear a ghost coloured according to its owner next to each avatar before translating to the new avatar. \textit{Popup notification} consists of displaying a notification pop-up associating each avatars to its new color.
	\item[Data:] 54 users, heterogeneous regarding VR knowledge. This group was divided into three parts. Each part was assigned to one of the three metaphors.
	\item[Evaluation criteria:] ranking of the three exchange metaphors over 10 criteria that we can group into two types: subjective pleasantness and objective performance (completion time of a task and error rate when identifying who is who).
	\item[Results:] the ranking analysis led to the conclusion that in term of pleasantness and understanding, \textit{Ghost} is the best metaphor, a bit better than \textit{Flickering} and far better than \textit{Popup notification}. On the contrary, \textit{Popup notification} led to the best mean of completion time. The three metaphors led to equivalent error rates and general ease of use.
\end{description}

\paragraph{Triggering exchange test}
\begin{description}
	\item[General description:] a user is placed in a virtual environment, controlling an avatar with a first person point of view. He changes his avatar with another one. The same system of color than for the first test is used.
	\item[Tested parameters:] 5 metaphors are used. Each combine a camera motion (\textit{straight}, \textit{curved} or \textit{none}) and a visual effect (\textit{ghost} or \textit{none}): (\textit{straight}, \textit{ghost}), (\textit{straight}, \textit{none}), (\textit{curved}, \textit{ghost}), (\textit{curved}, \textit{none}) and (\textit{none}, \textit{none}). The motion \textit{straight} means that the camera directly moves from the point of view of the original avatar to the new one. With the \textit{curved} motion, the camera starts and ends at the same points but takes a third person point of view while moving. The \textit{none} motion means that the point of view of the user changes instantaneously.
	\item[Data:] 42 users, heterogeneous regarding VR knowledge. They all saw and compared   the 5 metaphors.
	\item[Evaluation criteria:] ranking of the metaphors with two by two comparisons, and with marks given for 5 criteria related to the understanding degree and the pleasantness.
	\item[Results:] (\textit{none}, \textit{none}) is the best choice because it is easy to understand and simple. The \textit{straight} motions are preferred to the \textit{curved} ones, because they represent well the travel from an avatar to another. The \textit{ghost} visual effect is also appreciated for the same reason. The worst metaphor is (\textit{curved}, \textit{none}) because of the lack of information it carries and the disturbing camera motion.
\end{description}

\section{Show-Through Techniques}

\subsection{Introduction}
This paper presents solutions to the problem of showing hidden objects in a virtual collaborative environment. In the real world, one can walk and reach a point of view from which he is able to see the object. Nevertheless, in some situations, it is necessary to stand close to another person and so violate social protocols (each person have a zone around him where he won't accept the presence of the others). The proposed solutions consists of making obstacles disappear and so show through them.

\end{document}