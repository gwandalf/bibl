\documentclass[a4paper]{article}

\usepackage[english]{babel}
\usepackage[utf8]{inputenc}
\usepackage{amsmath}
\usepackage{graphicx}

\title{Your Paper}

\author{You}

\date{\today}

\begin{document}
\maketitle

\begin{abstract}
Your abstract.
\end{abstract}

\tableofcontents

\section{Introduction}

\section{State of the art}
Yet 7 pages !
For each subsection, make a synthesis of the indicated set of papers.
General remark: 

\subsection{Awareness}
%Presenting information with metaphors
Communication and awareness.

\subsubsection{Executor: awareness of his own actions}
Exchange of avatars.

\subsubsection{Awareness of a third person actions}
Show-Through techniques.
Exchange of avatars.

\subsection{Cooperation}

3-Hand manipulation.

\subsubsection{Particular case: using the VE as an informative support}
Communication and awareness (written communication).

\subsection{Social invasive aspects}

%Important and short (1p)

\subsubsection{Physical actor}
Show-Through techniques (social protocols).

\subsubsection{Intrusive metaphors}
Intrusive hand and metaphors, to be controlled

\subsection{Scenario design}
\# SEVEN.

\section{Goals}

The user will have to achieve a set of tasks. They will be the pretext to test different aspects of the coach presence and interaction.

\paragraph{Directives} In order to tell to the user what is his next task, the coach will have to communicate with him. Three communication modes will be evaluated:
\begin{enumerate}
	\item oral communication;
	\item instructions written or drawn on a virtual support object (e.g, a board or a tablet);
	\item instructions written or drawn in a virtual menu that the user can open as a popup.
\end{enumerate}

\paragraph{Invasive aspects} The potentially social invasive metaphors have to be evaluated. To achieve this goal, a use case will be the coach demonstration step necessary for complex tasks. Four ways will be evaluated:
\begin{enumerate}
	\item the coach takes-over the trainee avatar, who can only see what happens but can't interact any more;
	\item the coach uses his own avatar;
	\item the coach uses his own avatar but only the interacting body parts are visible (typically the hands);
	\item The coach uses his own avatar but is not visible at all.
\end{enumerate}

\paragraph{Collaboration} Some collaborative tasks can be a support for the evaluation of their efficiency with respect to the way the coach is represented. Three metaphors can be evaluated:
\begin{enumerate}
	\item the avatar of the coach is visible;
	\item only the interacting body parts are visible;
	\item the coach is invisible but his contact points with the handled objects are highlighted (e.g, as red spots).
\end{enumerate}

\section{Conclusion}

\end{document}