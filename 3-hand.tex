\documentclass[a4paper]{article}

\usepackage[english]{babel}
\usepackage[utf8]{inputenc}
\usepackage{amsmath}
\usepackage{graphicx}
\usepackage[colorinlistoftodos]{todonotes}

\title{Your Paper}

\author{You}

\date{\today}

\begin{document}
\maketitle

\begin{abstract}
Your abstract.
\end{abstract}

\section{3-Hand manipulation}

\subsection{Introduction}

This paper presents a tool that allows to 2 or 3 users to manipulate virtual the same virtual object simultaneously. The 6 freedom degrees can be reproduced.

\subsection{State of the art}

Multi-User interaction tools already exist, but they are limited and not representative of the real world interactions. Some aren't symetric: the DOF are distributed over the users, like "trackball". Others don't allow one of the DOF (the rotation around the axis defined by two handling hands) or at least don't reproduce it in a realistic way, like "SkeweR". By the way, the good ways to combine two interactions change from an application to another (ex : adding, compute the mean).

The determination of the sixth DOF is impossible because of the number of contact points, which is only 2. Therefore, it's impossible to generate forces able to make the object rotate around the defined axis.

\subsection{Solution}

The simple solution is to define a third point of handling. Now we have a plane and therefore a possible change of orientation. This rotation is computed as the combination of two (sufficients) rotations.

\section{Exchange of avatars}

\subsection{Introduction}

This paper presents and compares several ways to represent an exchange of avatars, for the actors populating the virtual environment to be aware of the exchange. Two cases are considered: if the user to be notified is involved in the exchange or not. One of the conclusions of the experiments is that it's not necessary to perform a visual feedback of the exchange even if it's more efficient.

\subsection{State of the art}



\section{Some \LaTeX{} Examples}
\label{sec:examples}

\subsection{How to Leave Comments}

Comments can be added to the margins of the document using the \todo{Here's a comment in the margin!} todo command, as shown in the example on the right. You can also add inline comments:

\todo[inline, color=green!40]{This is an inline comment.}

\subsection{How to Include Figures}

First you have to upload the image file (JPEG, PNG or PDF) from your computer to writeLaTeX using the upload link the project menu. Then use the includegraphics command to include it in your document. Use the figure environment and the caption command to add a number and a caption to your figure. See the code for Figure \ref{fig:frog} in this section for an example.

\begin{figure}
\centering
\includegraphics[width=0.3\textwidth]{frog.jpg}
\caption{\label{fig:frog}This frog was uploaded to writeLaTeX via the project menu.}
\end{figure}

\subsection{How to Make Tables}

Use the table and tabular commands for basic tables --- see Table~\ref{tab:widgets}, for example.

\begin{table}
\centering
\begin{tabular}{l|r}
Item & Quantity \\\hline
Widgets & 42 \\
Gadgets & 13
\end{tabular}
\caption{\label{tab:widgets}An example table.}
\end{table}

\subsection{How to Write Mathematics}

\LaTeX{} is great at typesetting mathematics. Let $X_1, X_2, \ldots, X_n$ be a sequence of independent and identically distributed random variables with $\text{E}[X_i] = \mu$ and $\text{Var}[X_i] = \sigma^2 < \infty$, and let
$$S_n = \frac{X_1 + X_2 + \cdots + X_n}{n}
      = \frac{1}{n}\sum_{i}^{n} X_i$$
denote their mean. Then as $n$ approaches infinity, the random variables $\sqrt{n}(S_n - \mu)$ converge in distribution to a normal $\mathcal{N}(0, \sigma^2)$.

\subsection{How to Make Sections and Subsections}

Use section and subsection commands to organize your document. \LaTeX{} handles all the formatting and numbering automatically. Use ref and label commands for cross-references.

\subsection{How to Make Lists}

You can make lists with automatic numbering \dots

\begin{enumerate}
\item Like this,
\item and like this.
\end{enumerate}
\dots or bullet points \dots
\begin{itemize}
\item Like this,
\item and like this.
\end{itemize}
\dots or with words and descriptions \dots
\begin{description}
\item[Word] Definition
\item[Concept] Explanation
\item[Idea] Text
\end{description}

We hope you find write\LaTeX\ useful, and please let us know if you have any feedback using the help menu above.

\end{document}