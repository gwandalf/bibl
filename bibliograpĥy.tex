\documentclass[11pt]{article}
\usepackage[french]{babel}
\usepackage[utf8]{inputenc}
\usepackage{fullpage}
\usepackage{graphicx}
\usepackage{subfigure}
\usepackage{comment}
\newcommand{\HRule}{\rule{\linewidth}{0.5mm}}


%numeroter les pages
\pagestyle{plain}


\begin{document}

\begin{titlepage}

\begin{center}

% Upper part of the page

 

\includegraphics[width=0.8\textwidth]{res/header}\\[1cm]
\textsc{\Large Master research Internship}
\vspace{1cm}


\includegraphics[width=0.09\textwidth]{res/insa-rennes}

  
\vspace{1cm} 
\textsc{\Large Bibliographic report }\\[0.5cm]


% The title of your report
\HRule \\[0.4cm]
{ \Large \bfseries Sailor vs Poseidon: Enhancing Training Simulation in 3D Collaborative Virtual Environments }\\[0.4cm]

\HRule \\[1.5cm]
% The domain of your research 
%\textbf{KEEP  WITHIN THIS LIST (see model.tex) ONE OR TWO DOMAIN(S) THAT CORRESPOND(S) TO %YOUR INTERNSHIP - COMMENT OR REMOVE ALL THE OTHER ONES -}
\begin{flushleft}
\textbf{Domain : Technology for Human Learning, Human-Computer Interaction }
\end{flushleft}

\begin{comment}
Technology for Human Learning 
Artificial Intelligence 
Computer Arithmetic
Hardware Architecture
Automatic Control Engineering
Bioinformatics 
Biotechnology
Computational Complexity 
Computational Engineering, Finance, and Science
Computational Geometry 
Computation and Language 
Cryptography and Security 
Computer Vision and Pattern Recognition
Computers and Society 
Databases 
Distributed, Parallel, and Cluster Computing 
Digital Libraries
Discrete Mathematics 
Data Structures and Algorithms 
Embedded Systems 
Emerging Technologies 
Formal Languages and Automata Theory 
General Literature 
Graphics 
Computer Science and Game Theory 
Human-Computer Interaction 
Computer Aided Engineering 
Medical Imaging 
Information Retrieval 
Information Theory 
Ubiquitous Computing 
Machine Learning
Logic in Computer Science 
Multiagent Systems 
Mobile Computing
Multimedia
Modeling and Simulation 
Mathematical Software 
Numerical Analysis 
Neural and Evolutionary Computing 
Networking and Internet Architecture 
Operating Systems 
Performance 
Programming Languages 
Robotics 
Operations Research
Symbolic Computation 
Sound
Software Engineering 
Social and Information Networks 
Systems and Control 
Image Processing 
Signal and Image Processing 
Document and Text Processing
Web
\end{comment}
%
% Author and supervisor(s)
\begin{minipage}{0.4\textwidth}
\begin{flushleft} \large
\emph{Author:}\\
Gwendal  \textsc{Le Moulec}
\end{flushleft}
\end{minipage}
\begin{minipage}{0.4\textwidth}
\begin{flushright} \large
\emph{Supervisors:} \\
%
% name(s) of your supervisor(s)
Valérie \textsc{Gouranton } \\
Thi Thuong \textsc{Huyen Nguyen } \\
Fernando \textsc{Argelaguet}\\
Anatole \textsc{L\'ecuyer}\\
% Name of your team
Hybrid
\end{flushright}
\end{minipage}

\vfill


% INCLUDE HERE THE LOGO OF YOUR INSTITUTION
%\textbf{INSERT ``\%'' IN FRONT OF ALL THE LOGO YOU DO NOT NEED - A SINGLE ONE SHOULD %REMAIN AT THE BOTTOM OF THIS PAGE}
\begin{flushleft}
%\includegraphics[width=0.09\textwidth]{./supelec}\\
\end{flushleft}
\begin{flushleft}
%\includegraphics[width=0.09\textwidth]{./logoUbs}
\end{flushleft}
\begin{flushleft}
%\includegraphics[width=0.09\textwidth]{./UBO}
\end{flushleft}
\begin{flushleft}
%\includegraphics[width=0.09\textwidth]{./tel-br}
\end{flushleft}
\begin{flushleft}
%\includegraphics[width=0.09\textwidth]{./rennes1}
\end{flushleft}
\begin{flushleft}
\includegraphics[width=0.09\textwidth]{res/insa-rennes}
\end{flushleft}
\begin{flushleft}
%\includegraphics[width=0.09\textwidth]{./esir}
\end{flushleft}
\begin{flushleft}
%\includegraphics[width=0.09\textwidth]{./enssat}
\end{flushleft}
\begin{flushleft}
%\includegraphics[width=0.09\textwidth]{./ENS-Rennes}
\end{flushleft}
\begin{flushleft}
%\includegraphics[width=0.09\textwidth]{./logo_ENIB}
\end{flushleft}
\end{center}
\end{titlepage}



%************************************************************%

\begin{abstract}
Here should appear your abstract - Should be 15 lines long 
\end{abstract}

% compile twice to get the table of contents
\tableofcontents
\newpage

\setcounter{page}{1} 

%*****************************************************************%
\section{Introduction}

Here start your document - Should be 15 pages long 

Please do not hesitate to have a look at the bibliographic report included in this archive 

Yet 7 pages !
For each subsection, make a synthesis of the indicated set of papers.
General remark: 

\section{Prise de conscience}

En EV collaboratif ou non, il est essentiel de prendre conscience des événements extérieurs, qu'ils soient directement liés aux tâches de l'utilisateur ou pas. La prise de conscience se fait principalement à l'aide de métaphores. On distingue notamment deux types de prises de conscience~: la prise de conscience d'événement ou de faits généraux qui ne sont pas directement liés à une tâche à réaliser en particulier et la prise de conscience destinée à faciliter un tâche donnée.

\subsection{Facilitation de tâches}
%Show-Through techniques.
%Exchange of avatars.
Lorsqu'un utilisateur doit effectuer une tâche précise, par exemple atteindre une zone cible, il peut être guidé par des métaphores. Ces indications ont pour but d'améliorer la performance d'un utilisateur ou de le diriger plus rapidement vers les points d'intérêt plutôt que de le laisser perdre du temps en cherchant par lui-même.
\\

Certains guidages ont un but purement didactique, tandis que d'autres servent 

\subsection{Prise de conscience d'événements}
La prise de conscience des événements se déroulant dans un EV à pour objectif de ne pas perdre les utilisateurs, qui doivent comprendre ce qu'il se passe autour d'eux. Quelques fois, cela peut même avoir une incidence sur les tâches à réaliser. Par exemple, les échanges d'avatars de l'article \cite{avatars} sont importants à prendre en compte pour l'utilisateur car il faut successivement demander de l'aide à des experts précis qui peuvent changer d'avatars plusieurs fois au cours de la simulation. Cependant, ce genre d'événements pouvant se produire n'importe quand et sans raison particulière, il ne peut pas être classé dans les facilitations de tâches car il n'est pas destiné à guider l'utilisateur.
\\

Les faits et événements peuvent être regroupés en différentes catégories (voir \cite{survey})~:
\begin{itemize}
	\item \textbf{les utilisateurs~:} il faut pouvoir retranscrire leurs activités. Pour cela, les supports des métaphores sont leurs avatars. Un cas d'usage serait le fait de mettre en évidence un avatar qui a un rôle particulier dans l'environnement, par le moyen d'une icône placée au-dessus de lui par exemple.
	\item \textbf{l'environnement virtuel~:} des objets ou des zones de la scène peuvent se caractériser par leur état. Dans certaines applications, certains objets sont manipulables et d'autres non. Il peut alors être intéressant de coder cette information par un code couleur.
	\item \textbf{les interactions~:} lorsqu'un utilisateur interagit avec un objet, il peut être utile de représenter cette information par un lien entre l'utilisateur et l'objet.
	\item \textbf{l'environnement physique~:} il ne faut pas oublier que les utilisateurs évoluent avant tout dans un environnement réel constitué d'obstacles, par exemple les murs dans un système CAVE ou les autres utilisateurs. A défaut de pouvoir éviter les collisions, Il faut au moins diminuer les risques. Une possibilité est d'afficher un message de prévention lorsqu'un utilisateur est trop prêt d'un mur, ou une barrière virtuelle destinée à dissuader l'utilisateur, comme présenté sur la figure \ref{fig:barrier}.
	\item \textbf{Les erreurs internes~:} certaines erreurs du système peuvent introduire des incohérences dans l'EV. C'est notamment le cas lors de problèmes de réseau pour un EVC~: si le système chargé de gérer un objet ou un avatar particulier ne communique plus de données, les interactions qu'auraient alors les utilisateurs avec cet avatar ou cet objet seraient incohérentes. Il peut y avoir aussi un déphasage entre ce que voient deux utilisateurs différents. Dans ces deux cas, il est important de faire prendre conscience de l'anomalie, en gelant l'objet par exemple.
\end{itemize}

\begin{figure}
\centering
\includegraphics[width=1\textwidth]{res/barrier}
\caption{\label{fig:barrier}Barrière de dissuasion pour éviter une collision avec le mur réel \cite{survey}.}
\end{figure}

Quelque soit le type de métaphore à mettre en œuvre, il faut toujours faire un compromis entre la qualité de la métaphore et son coût. La qualité d'une métaphore se mesure par sa capacité à transmettre une information correcte, compréhensible et complète, ainsi que par son influence sur le sentiment d'immersion éprouvé par les utilisateurs. Le coût inclut bien sûr la consommation de ressources, mais se caractérise surtout par la gêne que peut introduire une métaphore. Par exemple, dans le cas d'un objet vu sous différents états selon les utilisateurs à cause de problèmes liés au réseau, une solution consiste à faire apparaître à côté de l'objet un fantôme par état différent. Cependant, faire apparaître trop d'états peut encombrer l'espace, ce qui n'est pas souhaitable.
\\

Un exemple qui montre bien les différents niveaux de métaphores privilégiant soit certains critères de qualité, soit un faible coût est celui de l'échange d'avatars de \cite{avatars}. Une des expériences présentées consiste à tester différentes métaphores permettant de prendre conscience d'un échange d'avatars en tant que témoin extérieur à l'échange. Le témoin doit réaliser une séquence d'opérations de maintenance sur une voiture. Certaines opérations requièrent l'expérience d'un expert incarnant un avatar. Il y a plusieurs experts à appeler au cours de la simulation et ces derniers échangent régulièrement leurs avatars. Pour faciliter la compréhension, chaque expert est associé à une couleur. Trois métaphores différentes sont testées~:
\begin{enumerate}
	\item \textit{Flickering}~: chaque avatar impliqué dans l'échange clignote en prenant la couleur de l'expert qui le contrôle, puis en prenant la couleur du nouvel expert.
	\item \textit{Ghosts}~: un fantôme de la couleur de l'expert apparaît à côté de l'avatar d'origine et se déplace jusqu'à l'avatar de destination.
	\item \textit{Popup}~: une simple popup apparaît et indique quel échange a lieu.
\end{enumerate}

La vidéo \texttt{http://youtu.be/Yu4YpGzKXes} montre un exemple d'échange pour chaque métaphore. Les critères de qualité et de coût ont étés évalués par le moyen d'un questionnaire, sur une population de test composée de 54 participants divisé en trois groupes (un pour chaque métaphore). Chaque critère est évalué par une ou plusieurs affirmations (e.g "La métaphore est facile à comprendre" ou "La métaphore distrait de la tâche") notées par un degré de satisfaction allant de 1 à 5. les résultats révèlent entre autres que même si \textit{Ghosts} est jugée de bien meilleure qualité que \textit{Popup} sur tous les critères, \textit{Popup} est la métaphore qui donne le meilleur temps d'exécution des tâches. \textit{Ghosts} est aussi utilisée dans une autre expérience de l'article, qui compare des métaphores représentant un échange d'avatars dans lequel l'utilisateur est impliqué. Elle est alors considérée par une partie des utilisateurs comme étant trop distrayante, ce qui rentre dans les critères de coût.
\\

Les compromis à faire dépendent beaucoup du type d'événement ou de fait à représenter. Pour apporter des informations sur un objet de l'EV, une bonne solution consiste à jouer sur sa couleur ou sa transparence, car cela n'encombre pas l'EV et permet de bien identifier quel objet est concerné. La difficulté et alors de choisir une couleur ou un degré de transparence qui n'altère pas la perception de l'environnement. Par exemple, il n'est pas souhaitable de faire disparaître complètement un mur occultant une zone d'intérêt, car cela risque de faire oublier la présence du mur. Il vaut mieux en rendre un pan moyennement transparent. Dans le cas de la remontée d'erreurs internes ou d'événements de faible importance, il semble à priori plus souhaitable d'avoir recours à des popups discrètes plutôt que d'encombrer l'EV qui sert de support direct aux informations plus importantes.

\section{Coopération}

\subsection{Manipulation d'objets}
3-Hand manipulation.

\subsection{Supports d'information}
Communication and awareness (written communication).

\section{Social invasive aspects}

%Important and short (1p)

\subsection{Physical actor}
Show-Through techniques (social protocols).

\subsection{Intrusive metaphors}
Intrusive hand and metaphors, to be controlled

\section{Scenario design}
\# SEVEN.

\section{Goals}

The user will have to achieve a set of tasks. They will be the pretext for testing different aspects of the coach presence and actions. What we want to evaluate through the tests are in general the global execution performance of the trainee and to what degree the ways the coach presence is represented are pleasant or at the contrary disturbing.

\paragraph{Directives} In order to tell to the user what is his next task, the coach will have to communicate with him. Three communication modes will be evaluated:
\begin{enumerate}
	\item oral communication;
	\item instructions written or drawn on a virtual support object (e.g, a board or a tablet);
	\item instructions written or drawn in a virtual menu that the user can open as a popup.
\end{enumerate}
The goal here is to evaluate the correlation between the execution time and performance and the communication way used.

\paragraph{Invasive aspects} The potentially social invasive metaphors have to be evaluated. To achieve this goal, a use case will be the coach demonstration step necessary for complex tasks. Four ways will be evaluated:
\begin{enumerate}
	\item the coach takes-over the trainee avatar, who can only see what happens but can't interact any more;
	\item the coach uses his own avatar;
	\item the coach uses his own avatar but only the interacting body parts are visible (typically the hands);
	\item The coach uses his own avatar but is not visible at all, however his contact points with the handled objects are highlighted (e.g, as red spots).
\end{enumerate}
Through this type of tests, we want to evaluate the degree of annoyance of the used metaphors.

\paragraph{Collaboration} Some collaborative tasks can be a support for the evaluation of their efficiency and execution time with respect to the way the coach is represented. Three metaphors can be evaluated:
\begin{enumerate}
	\item the avatar of the coach is visible;
	\item only the interacting body parts are visible;
	\item the coach is invisible but his contact points with the handled objects are highlighted.
\end{enumerate}

\paragraph{Awareness of the coach presence and power} During the dead times, it can be interesting to evaluate the trainee perception of the coach actions on the VE. Two types of coach actions must be considered:
\begin{enumerate}
	\item the coach interacts with an object of the VE, e.g he changes the position of an object;
	\item the coach changes the parameters of the application, e.g he changes the weather conditions.
\end{enumerate}

For the first type of actions, The same metaphors than for the collaboration tasks can be used. For the second type, four metaphors can be evaluated:
\begin{enumerate}
	\item The coach avatar casts a spell using a specific gesture and voice control. Such a multimodal command can be real, i.e the coach has to perform it in order to make a parameter change, or just a metaphor, i.e the coach doesn't perform it but his avatar moves and speaks automatically.
	\item A 3D animation is performed by the system. For example, a grey cloud can appear and move through the scene in order to indicate that the weather is now rainy.
	\item A popup indicating the change is displayed.
	\item No indication at all.
\end{enumerate}

Several perception aspects can be evaluated thanks to these tests: the annoyance and pleasant degrees and the execution time (some metaphors can capture the attention of the trainee for a quite long time).

\section{Conclusion}

\begin{thebibliography}{1}
	\bibitem{avatars} LOPEZ, Thomas, BOUVILLE, Rozenn, LOUP-ESCANDE, Emilie, et al. \textit{Exchange of avatars: Toward a better perception and understanding. Visualization and Computer Graphics, IEEE Transactions on}, 2014, vol. 20, no 4, p. 644-653.
	\bibitem{survey} NGUYEN, Thi Thuong Huyen, DUVAL, Thierry, et al. A Survey of Communication and Awareness in Collaborative Virtual Environments. In : \textit{2014 International Workshop on Collaborative Virtual Environments (3DCVE)}. 2014.
	\bibitem{show-through} ARGELAGUET SANZ, Fernando, KUNERT, André, KULIK, Alexander, et al. Improving co-located collaboration with show-through techniques. 2010.
\end{thebibliography}

\end{document}